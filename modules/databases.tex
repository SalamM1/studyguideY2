\section{Databases and Sets}
This section deals with some definitions and how to formulate SQL Queries
\subsection{Definitions}
\paragraph{Relations}
\begin{itemize}
	\item Domain (D): An arbitrary (non-null) set of atomic values
	\item Attribute Name (A): A symbol with an associate domain
	\item Relational Schema (R): A finite set of attribute names
	\item Tuple (t): a mapping from the attributes of a relational schema to the union of their domains
	\item Relation (r): A finite set of tuples of relational schema
	\item Degree: Number of attributes [columns in an SQL table]
	\item Cardinality: Number of tuples [rows in an SQL table]
\end{itemize}
\paragraph{Keys}
\begin{itemize}
	\item Superkey: a set of attributes that can \textit{always} be used to differentiate one tuple from another
	\item Key - a minimal superkey
	\item Concatenated Key -  a key with more than one attribute
	\item Candidate Key - any key
	\item \textbf{Primary Key} - one of the candidate keys used to reference from other relations
	\item \textbf{Foreign Key} - an attribute of the relation which is a key for another relation
\end{itemize}
\paragraph{Data Types}
\begin{itemize}
	\item BOOLEAN: TRUE, FALSE, or NULL
	\item Strings: 
	\begin{itemize}
		\item CHAR(length)
		\item CHARACTER(length)
		\item VARCHAR
		\item TEXT
	\end{itemize}
	\item INTEGER
	\item REAL: A floating point
	\item Arbitrary Precision:
		\begin{itemize}
			\item NUMERIC
			\item DECIMAL
			\item MONEY
		\end{itemize}
	\item Date and Time
		\begin{itemize}
			\item TIMESTAMP
			\item DATE
			\item TIME
		\end{itemize}
\end{itemize}
\paragraph{Constraints}
\begin{itemize}
	\item DOMAIN: The domain/type of the attributes; eg INTEGER, VARCHAR
	\item NOT NULL: Ensures the column will not have a null value
	\item UNIQUE: Ensures uniqueness
	\item PRIMARY KEY: Not-null, unique column used as a primary identifier of a row
	\item FOREIGN KEY: Unique identifier to a row from another table
	\item CHECK(condition): Ensures all values in the column meet the defined condition (eg. cost $>$ 10)
	\item DEFAULT(default): Sets the value to the given value if no other value is specified on creation
\end{itemize}
\paragraph{Foreign Key Syntax}
\begin{itemize}
	\item ON DELETE (action): Does the specified action when the key or referenced table is deleted
	\item ON UPDATE (action): Does the specified action when the key or referenced table is updated
	\item NO ACTION: Action - Does nothing
	\item RESTRICT: Action - will not allow user to delete/update the key/referenced table
	\item CASCADE: Action - cascades the change on the original to the value found in the foreign table
	\item SET NULL: Action - sets the value to null in the foreign table
	\item SET DEFAULT: Action - sets the value to default in the foreign table
\end{itemize}

\subsection{SQL Statements}
\paragraph{CREATE}
\begin{verbatim}
CREATE TABLE tableName (
    attributeName TYPE CONTRAINTS,
    attributeName2 TYPE CONTRAINTS,
    exampleID TYPE CONSTRAINTS,
    exampleID2 TYPE CONSTRAINTS,

    PRIMARY KEY (exampleID),
    FOREIGN KEY (exampleID2) REFERENCES  otherTable(id)
        ON UPDATE action
        ON DELETE action;
)
\end{verbatim}

\paragraph{INSERT}
\begin{verbatim}
INSERT INTO tableName VALUES  (valueA, valueB, valueC, ...);

INSERT INTO tableName(attributeA, attributeB, attributeC) 
VALUES  (valueA, valueB, valueC);
\end{verbatim}

\paragraph{UPDATE}
\begin{verbatim}
UPDATE tableName SET attributeA = valueA WHERE id = valueID;
\end{verbatim}

\paragraph{DELETE}
\begin{verbatim}
	DELETE FROM tableName WHERE attributeA = valueA;
\end{verbatim}

\paragraph{SELECT}
\begin{verbatim}
	SELECT * FROM tableName WHERE condition
\end{verbatim}
\section{ER Diagrams}
\section{JDBC}
\section{Concurrency}