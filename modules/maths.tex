\section{Vectors and Matrices}
This section deals with operations on vectors and matrices that are commonly used.
\subsection{Vector Operations}
	\textbf{Magnitude}
	\begin{equation}
	\left\|   \vec{V} \right\|  = \sqrt{\sum_{i=1}^{n} v_i^2}
	\end{equation}
	\textbf{Dot Product}
	\begin{equation}
	\vec{V} \cdot \vec{U} = \sum_{i=1}^{n} v_iu_i
	\end{equation}
	\textbf{3x3 Cross Product}
\begin{equation}
	\vec{V} \times \vec{U} = \begin{bmatrix}
		v_2u_3 - v_3u_2\\ 
		v_3u_1 - v_1u_3\\ 
		v_1u_2 - v_2u_1
	\end{bmatrix}
\end{equation}
	\textbf{Equations with angles}
		\begin{equation}
		\cos{\theta} = \frac{\vec{V} \cdot \vec{U}}{\left\| \vec{V} \right\| \left\| \vec{U} \right\| }
		\end{equation}
		\begin{equation}
		\sin{\theta} = \frac{\vec{V} \times \vec{U}}{\left\| \vec{V} \right\| \left\| \vec{U} \right\| }
		\end{equation}
\subsection{Matrix Operations}
\textbf{Multiply}
\begin{equation}
a x b = coming soon
\end{equation}
\textbf{2x2 Determinant}
\textbf{3x3 Determinant}
\textbf{Row Operations}
\textbf{Inverting a Matrix}
\newpage
\section{Fields}
This section deals with the little we need to know about Galois Fields, primarily GF(2).
\subsection{Notes}
\subsection{Proofs}
\newpage
\section{Lines and Planes}
This section covers everything done regarding systems of equations regarding lines and planes.
\subsection{Notes}
\subsection{Examples}
\newpage
\section{Set Theory}
This section covers the basics of Set Theory, with some examples.
\subsection{Notes}
\subsection{Examples}
\newpage
\section{Functions}
Expanding from Set Theory, this section deals with Functions and our new understanding of them.
\subsection{Notes}
\subsection{Examples}
\newpage
\section{Probability}
Continuing from Year 1 AI, probability.
\subsection{Notes}
\subsection{Examples}
\newpage
\section{Random Variables}
This section deals with the handling of both discrete and continuous random variables. Don't go gambling after you cover this section.
\subsection{Discrete}
\subsection{Continuous}
\subsection{Examples}
\newpage
\section{Equations}
Appending-like area to store all equations that will likely be used in the exam.
